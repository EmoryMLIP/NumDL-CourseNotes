\documentclass[12pt,fleqn,handout]{beamer}


\xdefinecolor{lavendar}{rgb}{0.8,0.6,1}
\xdefinecolor{olive}{cmyk}{0.64,0,0.95,0.4}
%\xdefinecolor{olive}{cmyk}{1,0,0,0}
\xdefinecolor{mag}{cmyk}{0.1,1,0,0.2}
\xdefinecolor{lblue}{rgb}{0,0,1.5}
\xdefinecolor{lred}{rgb}{1,0,0}
\xdefinecolor{mine}{cmyk}{1,0,0.2,0}
\xdefinecolor{bluel}{cmyk}{0.1,0,0.9,0.4}

\usepackage{amsmath,amssymb,dsfont,mathrsfs}
\usepackage{tikz,pgflibraryplotmarks}
\usepackage{multimedia}
\usepackage{wasysym}
\usepackage{rotating}
\usepackage{algorithm,algorithmic}
\usepackage{graphicx} % more modern
\usepackage{subfigure}
\usepackage{booktabs}

\usepackage{pgfplots}
\usepackage{verbatim}

\usepackage{setspace}
\newlength\iwidth
\newlength\iheight

\newcommand\makebeamertitle{\frame{\maketitle}}%
\graphicspath{{./images/}}
\setbeamertemplate{navigation symbols}{}
\addtobeamertemplate{navigation symbols}{}{%
    \usebeamerfont{footline}%
    \usebeamercolor[fg]{footline}%
	\insertshorttitle
    \;--
    \insertframenumber
}

\newcommand{\sectionstart}{
	\only<beamer>{
 	\begin{frame}% (fold)
 		\begin{centering}\Huge \insertsection \par\end{centering}
 	\end{frame}% frame the_application (end)
	}
 }


% make bibliography entries smaller
\usepackage{natbib}
\setbeamertemplate{bibliography item}{[\theenumiv]}
\renewcommand\bibfont{\scriptsize}
\setbeamertemplate{frametitle continuation}[from second]
\newcommand{\tcr}{\textcolor{red}}
\newcommand{\tcrd}{\textcolor{red}}
\newcommand{\tcb}{\textcolor{bluel}}
\newcommand{\tcm}{\textcolor{mag}}
\newcommand{\tcg}{\textcolor{olive}}

\newcommand{\R}{\mathbb{R}}
\newcommand{\C}{\mathbb{C}}

% bold lower-case for vectors
\newcommand{\bfa}{{\bf a}}
\newcommand{\bfb}{{\bf b}}
\newcommand{\bfc}{{\bf c}}
\newcommand{\bfs}{{\bf s}}
\newcommand{\bfm}{{\bf m}}
\newcommand{\bfd}{{\bf d}}
\newcommand{\bfe}{{\bf e}}
\newcommand{\bfu}{{\bf u}}
\newcommand{\bfy}{{\bf y}}
\newcommand{\bfx}{{\bf x}}
\newcommand{\bfh}{{\bf h}}
\newcommand{\bfw}{{\bf w}}
\newcommand{\bfv}{{\bf v}}
\newcommand{\bfr}{{\bf r}}
\newcommand{\bfz}{{\bf z}}
\newcommand{\bfp}{{\bf p}}


% bold upper-case for linear operators
\newcommand{\bfA}{{\bf A}}
\newcommand{\bfB}{{\bf B}}
\newcommand{\bfZ}{{\bf Z}}
\newcommand{\bfM}{{\bf M}}
\newcommand{\bfC}{{\bf C}}
\newcommand{\bfD}{{\bf D}}
\newcommand{\bfQ}{{\bf Q}}
\newcommand{\bfJ}{{\bf J}}
\newcommand{\bfG}{{\bf G}}
\newcommand{\bfI}{{\bf I}}
\newcommand{\bfP}{{\bf P}}
\newcommand{\bfK}{{\bf K}}
\newcommand{\bfY}{{\bf Y}}
\newcommand{\bfW}{{\bf W}}
\newcommand{\bfR}{{\bf R}}
\newcommand{\bfL}{{\bf L}}
\newcommand{\bfF}{{\bf F}}
\newcommand{\bfT}{{\bf T}}
\newcommand{\bfS}{{\bf S}}
\newcommand{\bfX}{{\bf X}}
\newcommand{\bfU}{{\bf U}}
\newcommand{\bfV}{{\bf V}}
\newcommand{\bfH}{{\bf H}}


\newcommand{\calF}{\mathcal{F}}



\newcommand{\hf}{{\frac 12}}
\newcommand{\bftheta}{{\boldsymbol \theta}}
\newcommand{\bfxi}{{\boldsymbol \xi}}

\newcommand{\bfLambda}{{\boldsymbol \Lambda}}
\newcommand{\bfSigma}{{\boldsymbol \Sigma}}
\newcommand{\bfepsilon}{{\boldsymbol \epsilon}}

\newcommand{\E}{\vec E}
\newcommand{\B}{\vec B}

\newcommand{\vu}{  {\vec {\bf u}}}

\newcommand{\grad}{  {\vec {\bf \nabla}}}

\newcommand{\lfrownie}{\textcolor{red}{\large{\frownie}}}
\newcommand{\lsmiley}{\textcolor{green}{\large{\smiley}}}

\newcommand{\curl}{\ensuremath{\nabla\times\,}}
\renewcommand{\div}{\nabla\cdot\,}
\newcommand{\divh}{\nabla_h\cdot\,}
\renewcommand{\grad}{\ensuremath{\nabla}}

\DeclareMathOperator*{\argmin}{arg\,min}


\title[Newton Softmax]{Classification using Newton's Method}
\subtitle{Numerical Methods for Deep Learning}
 \date{}
\begin{document}

\makebeamertitle

\section{Intro to Newton's Method} % (fold)
\label{sec:logistic_regression_and_softmax}

\begin{frame}\frametitle{Newton-like Methods}

Goal: Solve $\min_{\bfW} E(\bfW)$. Consider $k$th iteration. Assume $E$ convex.

\bigskip
\pause

To find optimal step $\bfD$, use Taylor's theorem
{\small{
$$ E(\bfW_k + \bfD) = E(\bfW_k) + \grad E(\bfW_k)^{\top} \bfD +
\hf \bfD^{\top} \grad^2 E(\bfW_k) \bfD +
{\cal O}(\|\bfD\|^3)  $$}}
and differentiate w.r.t $\bfD$ 
\pause
to obtain
$$ \grad^2 E(\bfW_k) \bfD = -\grad E(\bfW_k). $$

\pause

Practical Newton methods (see, e.g., \cite[Ch.7]{NocedalWright2006})
\begin{itemize}
\item do not compute $\bfD$ accurately (add line search for safety)
\item use, e.g., Conjugate Gradient (CG) methods
\item do not generate $\grad^2 E$ since CG only needs mat-vecs
\item give quadratic/superlinear/good linear convergence
\end{itemize}

\end{frame}

\begin{frame}[fragile]\frametitle{Newton-like Methods for Softmax}

Need to compute Hessian $\nabla^2 E$. Recall: 
\begin{align*}
	\grad_{\bfW}E &= \frac{1}{n} \left( -\bfC_{\rm obs} + \exp(\bfS) \odot 
	\left( \bfe_{n_c} \left({\frac {1}{\bfe_{n_c}^\top \exp(\bfS)}}\right)\right) \right)\bfY^\top\\
              & =  \grad_\bfS E(\bfS) \bfY^\top,
\end{align*}
where $\bfS = \bfW \bfY$. 
\pause
For Hessian we know 
$$
\nabla^2_\bfW E(\bfW) = \bfY \nabla_S^2 E(\bfS) \bfY^\top 
$$

\pause

Remarks:
\begin{itemize}
\item size of $\nabla^2_\bfS E$ is $n_c n \times  n_c n$, typically sparse
\item size of $\nabla^2_\bfW E$ is $n_c n_f \times n_c n_f$, typically dense
\item building Hessian can be costly (when $n$ is large)
\item Hessian is spd since $E$ is convex in $\bfS$
\end{itemize}


\end{frame}

\begin{frame}[fragile]\frametitle{Hessian of Softmax Function - 1}

Recall
$$ \grad_\bfS E = \frac{1}{n} \left(-\bfC + \exp(\bfS)  \odot \frac 1 {\bfe_{n_c}\bfe_{n_c}^\top\exp(\bfS)} \right) $$

Let's first vectorize this  $\bfs = {\rm vec}(\bfS)$ and $\bfc = {\rm vec}(\bfC)$

$$ \grad_\bfs E = \frac{1}{n} \left(- \bfc + \exp(\bfs)  \odot \frac{1 }{(\bfI \otimes (\bfe_{n_c} \bfe_{n_c}^{\top}))\exp(\bfs)} \right) $$

\bigskip
\pause

Use product rule

\begin{eqnarray*}
\nabla^2_\bfs E &=& {\rm diag} \left( {\frac 1 {(\bfI \otimes (\bfe_{n_c} \bfe_{n_c}^{\top})) \exp(\bfs)}} \right)
\bfJ_\bfs  \exp(\bfs)   +  \\
&&
{\rm diag} (\exp (\bfs)) \bfJ_\bfs \left( {\frac 1 {(\bfI \otimes (\bfe_{n_c} \bfe_{n_c}^{\top})) \exp(\bfs)}} \right)\\
&=& \nabla^2_\bfs E_1 + \nabla^2_\bfs E_2
\end{eqnarray*}
\end{frame}


\begin{frame}[fragile]\frametitle{Hessian of Softmax Function - 2}

First term is easy
\begin{eqnarray*}
\nabla^2_{\bfs} E_1 &=& {\rm diag} \left( {\frac 1 {(\bfI\otimes (\bfe_{n_c} \bfe_{n_c}^{\top})) \exp(\bfs)}} \right)
{\rm diag} \left( \exp(\bfs)\right) \\
&=&
{\rm diag} \left( {\frac {\exp(\bfs)} {(\bfI \otimes (\bfe_{n_c} \bfe_{n_c}^{\top})) \exp(\bfs)}} \right) 
\end{eqnarray*}
\pause
\bigskip


Reshaped back, a matrix-vector-product with $\bfV \in \R^{n_c \times n_f}$ is

$$ \bfH_1 \bfV =   \left( \left( {\frac {\exp(\bfS)} {\bfe_{n_c}\bfe_{n_c}^\top \exp(\bfS)}} \right) \odot ( \bfV \bfY) \right) \bfY^{\top} $$

\end{frame}

\begin{frame}[fragile]\frametitle{Hessian of Softmax Function - 3}
	
$$
E_2 = {\rm diag} (\exp (\bfs)) \underbrace{\bfJ_\bfs \left( {\frac 1 {(\bfI \otimes (\bfe_{n_c} \bfe_{n_c}^{\top})) \exp(\bfs)}} \right)}_{=:\bfT}.
$$
 Using chain rule, we get
 $$
 \bfT = - {\rm diag}\left({\frac 1 {\left((\bfI \otimes (\bfe_{n_c} \bfe_{n_c}^{\top} )) \exp(\bfs)}\right)^2}\right) (\bfI \otimes (\bfe_{n_c} \bfe_{n_c}^\top)) {\rm diag}(\exp(s))
 $$

 \pause\bigskip

 After reshape the matrix-vector-product with $\bfV \in \R^{n_f \times n_c}$ is

 \begin{align*}
  \bfH_2 \bfV=  -\left( \frac {(\exp(\bfS))}{\bfe_{n_c}(\bfe_{n_c}^\top \exp(\bfS))^2 }\right) 
   \odot (\bfe_{n_c}\bfe_{n_c}^\top (\exp(\bfS) \odot (\bfV \bfY)))\bfY^{\top}  
 \end{align*}

\end{frame}


\begin{frame}[fragile]\frametitle{Newton-CG for Softmax function }

Mat-vecs with Hessian can be computed as
\begin{eqnarray*}
	\nabla^2_\bfW E(\bfW) \bfV=& \frac{1}{n}  \left( \left( {\frac {\exp(\bfS)} {\bfe_{n_c}\bfe_{n_c}^\top \exp(\bfS)}} \right) \odot ( \bfV \bfY) \right)\bfY^\top \\
	-&\frac{1}{n}  \left( {\frac {(\exp(\bfS))}{\bfe_{n_c}(\bfe_{n_c}^\top\exp(\bfS))^2 }}\right)  \odot (\bfe_{n_c}\bfe_{n_c}^\top (\exp(\bfS) \odot (\bfV \bfY)) )\bfY^\top
\end{eqnarray*}
(possible to further simplify this to reduce operations)

\bigskip
\pause

Now, ready to use matrix-free Newton method with Armijo linesearch and CG solver that computes
$$
	\nabla^2_\bfW E(\bfW) \bfD \approx -\nabla_\bfW E(\bfW).
$$

Remarks:
\begin{itemize}
	\item how well to solve? use large tolerance on relative residual
	\item can accelerate CG with preconditioning $\leadsto$ PCG
	\item possible to omit second term in Hessian?
\end{itemize}
\end{frame}

\begin{frame}[fragile]\frametitle{Coding: Hessian of Softmax Function}

Extend your softmax function, so that it returns a function handle computing mat-vecs with Hessian if needed.

\begin{verbatim}
function[E,dE,d2Emv] = softmaxFun(W,Y,C)

% Your code from before
if nargout > 1
% Your code from before
end

if nargout > 2

% Your new code here
d2Emv = @(V) ...;
end
end
\end{verbatim}
\begin{center}
	Don't forget to check your derivatives!
\end{center}
\end{frame}


\begin{frame}\frametitle{Newton-like Methods - Derivatives}

Consider the softmax function

$$ E(\bfW) = -\sum \bfY \odot (\bfX \bfW) +\sum \log\left (\sum \exp(\bfX \bfW) \right) $$


{\bf Class problems:}

\begin{enumerate}
\item Compute the Hessian of the cross entropy function

\item Write code that constructs the matrix it and do a derivative check at a random point $\bfW_0$.

\item
Write a code that performs matrix vector products with the Hessian (without constructing it). Test by comparing results with the matrix-based code for a random vector.

\end{enumerate}


\end{frame}

\begin{frame}[allowframebreaks]
	\frametitle{References}
\bibliographystyle{abbrv}
 % \bibliographystyle{plainnat}
\bibliography{NumDNN}

\end{frame}

\end{document}
















